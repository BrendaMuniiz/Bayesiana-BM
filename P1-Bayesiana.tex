% Options for packages loaded elsewhere
\PassOptionsToPackage{unicode}{hyperref}
\PassOptionsToPackage{hyphens}{url}
%
\documentclass[
]{article}
\usepackage{lmodern}
\usepackage{amssymb,amsmath}
\usepackage{ifxetex,ifluatex}
\ifnum 0\ifxetex 1\fi\ifluatex 1\fi=0 % if pdftex
  \usepackage[T1]{fontenc}
  \usepackage[utf8]{inputenc}
  \usepackage{textcomp} % provide euro and other symbols
\else % if luatex or xetex
  \usepackage{unicode-math}
  \defaultfontfeatures{Scale=MatchLowercase}
  \defaultfontfeatures[\rmfamily]{Ligatures=TeX,Scale=1}
\fi
% Use upquote if available, for straight quotes in verbatim environments
\IfFileExists{upquote.sty}{\usepackage{upquote}}{}
\IfFileExists{microtype.sty}{% use microtype if available
  \usepackage[]{microtype}
  \UseMicrotypeSet[protrusion]{basicmath} % disable protrusion for tt fonts
}{}
\makeatletter
\@ifundefined{KOMAClassName}{% if non-KOMA class
  \IfFileExists{parskip.sty}{%
    \usepackage{parskip}
  }{% else
    \setlength{\parindent}{0pt}
    \setlength{\parskip}{6pt plus 2pt minus 1pt}}
}{% if KOMA class
  \KOMAoptions{parskip=half}}
\makeatother
\usepackage{xcolor}
\IfFileExists{xurl.sty}{\usepackage{xurl}}{} % add URL line breaks if available
\IfFileExists{bookmark.sty}{\usepackage{bookmark}}{\usepackage{hyperref}}
\hypersetup{
  pdftitle={Bayesiana P1},
  pdfauthor={Brenda da Silva Muniz 11811603; Mônica Amaral Novelli 11810453},
  hidelinks,
  pdfcreator={LaTeX via pandoc}}
\urlstyle{same} % disable monospaced font for URLs
\usepackage[margin=1in]{geometry}
\usepackage{color}
\usepackage{fancyvrb}
\newcommand{\VerbBar}{|}
\newcommand{\VERB}{\Verb[commandchars=\\\{\}]}
\DefineVerbatimEnvironment{Highlighting}{Verbatim}{commandchars=\\\{\}}
% Add ',fontsize=\small' for more characters per line
\usepackage{framed}
\definecolor{shadecolor}{RGB}{248,248,248}
\newenvironment{Shaded}{\begin{snugshade}}{\end{snugshade}}
\newcommand{\AlertTok}[1]{\textcolor[rgb]{0.94,0.16,0.16}{#1}}
\newcommand{\AnnotationTok}[1]{\textcolor[rgb]{0.56,0.35,0.01}{\textbf{\textit{#1}}}}
\newcommand{\AttributeTok}[1]{\textcolor[rgb]{0.77,0.63,0.00}{#1}}
\newcommand{\BaseNTok}[1]{\textcolor[rgb]{0.00,0.00,0.81}{#1}}
\newcommand{\BuiltInTok}[1]{#1}
\newcommand{\CharTok}[1]{\textcolor[rgb]{0.31,0.60,0.02}{#1}}
\newcommand{\CommentTok}[1]{\textcolor[rgb]{0.56,0.35,0.01}{\textit{#1}}}
\newcommand{\CommentVarTok}[1]{\textcolor[rgb]{0.56,0.35,0.01}{\textbf{\textit{#1}}}}
\newcommand{\ConstantTok}[1]{\textcolor[rgb]{0.00,0.00,0.00}{#1}}
\newcommand{\ControlFlowTok}[1]{\textcolor[rgb]{0.13,0.29,0.53}{\textbf{#1}}}
\newcommand{\DataTypeTok}[1]{\textcolor[rgb]{0.13,0.29,0.53}{#1}}
\newcommand{\DecValTok}[1]{\textcolor[rgb]{0.00,0.00,0.81}{#1}}
\newcommand{\DocumentationTok}[1]{\textcolor[rgb]{0.56,0.35,0.01}{\textbf{\textit{#1}}}}
\newcommand{\ErrorTok}[1]{\textcolor[rgb]{0.64,0.00,0.00}{\textbf{#1}}}
\newcommand{\ExtensionTok}[1]{#1}
\newcommand{\FloatTok}[1]{\textcolor[rgb]{0.00,0.00,0.81}{#1}}
\newcommand{\FunctionTok}[1]{\textcolor[rgb]{0.00,0.00,0.00}{#1}}
\newcommand{\ImportTok}[1]{#1}
\newcommand{\InformationTok}[1]{\textcolor[rgb]{0.56,0.35,0.01}{\textbf{\textit{#1}}}}
\newcommand{\KeywordTok}[1]{\textcolor[rgb]{0.13,0.29,0.53}{\textbf{#1}}}
\newcommand{\NormalTok}[1]{#1}
\newcommand{\OperatorTok}[1]{\textcolor[rgb]{0.81,0.36,0.00}{\textbf{#1}}}
\newcommand{\OtherTok}[1]{\textcolor[rgb]{0.56,0.35,0.01}{#1}}
\newcommand{\PreprocessorTok}[1]{\textcolor[rgb]{0.56,0.35,0.01}{\textit{#1}}}
\newcommand{\RegionMarkerTok}[1]{#1}
\newcommand{\SpecialCharTok}[1]{\textcolor[rgb]{0.00,0.00,0.00}{#1}}
\newcommand{\SpecialStringTok}[1]{\textcolor[rgb]{0.31,0.60,0.02}{#1}}
\newcommand{\StringTok}[1]{\textcolor[rgb]{0.31,0.60,0.02}{#1}}
\newcommand{\VariableTok}[1]{\textcolor[rgb]{0.00,0.00,0.00}{#1}}
\newcommand{\VerbatimStringTok}[1]{\textcolor[rgb]{0.31,0.60,0.02}{#1}}
\newcommand{\WarningTok}[1]{\textcolor[rgb]{0.56,0.35,0.01}{\textbf{\textit{#1}}}}
\usepackage{graphicx,grffile}
\makeatletter
\def\maxwidth{\ifdim\Gin@nat@width>\linewidth\linewidth\else\Gin@nat@width\fi}
\def\maxheight{\ifdim\Gin@nat@height>\textheight\textheight\else\Gin@nat@height\fi}
\makeatother
% Scale images if necessary, so that they will not overflow the page
% margins by default, and it is still possible to overwrite the defaults
% using explicit options in \includegraphics[width, height, ...]{}
\setkeys{Gin}{width=\maxwidth,height=\maxheight,keepaspectratio}
% Set default figure placement to htbp
\makeatletter
\def\fps@figure{htbp}
\makeatother
\setlength{\emergencystretch}{3em} % prevent overfull lines
\providecommand{\tightlist}{%
  \setlength{\itemsep}{0pt}\setlength{\parskip}{0pt}}
\setcounter{secnumdepth}{-\maxdimen} % remove section numbering

\title{Bayesiana P1}
\author{Brenda da Silva Muniz 11811603 \and Mônica Amaral Novelli 11810453}
\date{Setembro 2021}

\begin{document}
\maketitle

\hypertarget{problema-dado}{%
\section{Problema dado}\label{problema-dado}}

Considera-se um conjunto histórico de dados de densidade do solo em uma
região registrados por Henry Cavendish no século XVIII. Agora, supõe-se
que, de experimentos e medições prévios, a priori para θ, densidade da
terra, é considerada ser N(5,4; 0,01). O pesquisador registrou 23
medidas da densidade do solo. Para estes dados, temos a média de y =
5,48 e supõe-se aqui que a variância de seu erro de medida é conhecida e
igual a 0,04. Então, temos que a posteriori é simplesmente obtida
através da fórmula acima como sendo θ\textbar y ∼ N(5,46; 0,00303).

\hypertarget{priori-informativa}{%
\subsection{Priori informativa}\label{priori-informativa}}

Preencher aqui

\hypertarget{priori-nuxe3o-informativa-de-jeffreys}{%
\subsection{Priori não informativa de
Jeffreys}\label{priori-nuxe3o-informativa-de-jeffreys}}

Temos que distribuição não informativa de Jeffreys é dada por:

\(p(θ) ∝ [I(θ)]^{1/2}\)

Em que \(I(\theta)\) é a medida de informação esperada de Fisher de θ
através de X que é definida como:

\(I(θ) = E[-\frac{∂^2log p(x|\theta)}{∂\theta^2}]\)

\hypertarget{verossimilhanuxe7a}{%
\subsection{Verossimilhança}\label{verossimilhanuxe7a}}

Dados \(X_1, X_2, . . . , X_n\) amostra aleatória \(X ∼ N(μ, σ^2)\)

Temos que a função de densidade \(f_X(x)\) é dada por:

\(f_X(x) = \frac{1}{√2πσ^2}\exp({-\frac{1}{2σ^2}(x − μ)})\)

E nossa função de distribuição conjunta
\(f_{X_1,X_2,...,X_n}(x_1, x_2, . . . , x_n)\)

\(\prod_{i = 1}^{n} f_X(x_i) = \prod_{i=1}^{n} \frac{1}{\sqrt{2\piσ^2}}\exp(-\frac{1}{{2σ^2}}(x_i − μ)^2\)

Para uma amostra de tamanho n, a função de verossimilhança pode ser
escrita como:

\(l(θ; x) = (2πσ^2)^{−n/2}\exp{\frac{−1}{2σ^2} \sum_{i=1}^{n} (x_i − θ)^2}∝ \exp{\frac{−n}{2σ^2} (\overline{x}− θ)^2}\)

\hypertarget{posteriori}{%
\subsection{Posteriori}\label{posteriori}}

A distribuição a posteriori de θ dado x é \(N(µ_1, {τ_1}^2)\) sendo

\(µ1 = {\frac{{τ_0}^{−2}µ_0 + nσ^{−2}\overline{x}}{{τ_0}^{−2}+ nσ^{−2}}}\)
e \({τ_1}^{−2} = {τ_0}^{-2} + nσ^{−2}.\)

\hypertarget{resumo-a-posteriori-com-a-priori-informativa}{%
\subsection{Resumo a posteriori com a priori
informativa}\label{resumo-a-posteriori-com-a-priori-informativa}}

Preencher aqui

\hypertarget{resumo-a-posteriori-com-a-priori-nuxe3o-informativa}{%
\subsection{Resumo a posteriori com a priori não
informativa}\label{resumo-a-posteriori-com-a-priori-nuxe3o-informativa}}

Preencher aqui

\begin{Shaded}
\begin{Highlighting}[]
\KeywordTok{summary}\NormalTok{(cars) }\CommentTok{# teste}
\end{Highlighting}
\end{Shaded}

\begin{verbatim}
##      speed           dist       
##  Min.   : 4.0   Min.   :  2.00  
##  1st Qu.:12.0   1st Qu.: 26.00  
##  Median :15.0   Median : 36.00  
##  Mean   :15.4   Mean   : 42.98  
##  3rd Qu.:19.0   3rd Qu.: 56.00  
##  Max.   :25.0   Max.   :120.00
\end{verbatim}

\end{document}
